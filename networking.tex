\documentclass[a4paper,11pt]{paper}
\usepackage[utf8]{inputenc}
\usepackage[italian]{babel}
\usepackage{graphicx}
\usepackage[colorlinks=true,linkcolor=blue]{hyperref}
\graphicspath{./images/}

\def\code#1{\texttt{#1}}

\title{Networking}

\begin{document}

\maketitle
\tableofcontents
\newpage

\subsection{Introduzione}
Gioara
\subsubsection{Il modello ISO OSI}
\subsubsection{Internet protocol suit (TCP/IP)}

\newpage

\section{Livello Fisico}
Arianna

\subsection{Terminologia}
\subsubsection{Informazione}
\subsubsection{Codice}
\subsubsection{Segnale}
\subsubsection{Lunghezza d'onda}
\subsubsection{Spettro}
\subsubsection{Banda}

\subsection{Filtri}

\subsection{Flusso di trasmissione}
\subsubsection{Simplex}
\subsubsection{Half-Duplex}
\subsubsection{Full-Duplex}

\subsection{Modulazione}
\subsubsection{Ad onda continua}
\subsubsection{Impulsiva}
\subsubsection{Digitale}

\subsection{Qualità delle trasmissioni} non so se mi piace qui
\subsubsection{Ritardo}
\subsubsection{Tempo di risposta}
\subsubsection{Throughput}
\subsubsection{Latenza}
\subsubsection{Jitter}

\subsection{Alterazioni del segnale}
\subsubsection{Attenuazione}
\subsubsection{Distorzione}
\subsubsection{Rumore}
\subsubsection{Interferenza}

\subsection{Limiti alla velocità di trasferimento}
\subsubsection{Classificazione dei canali trasmissivi}
\subsubsection{Teorema di Nyquist}
\subsubsection{Teorema di Shannon}
\subsubsection{Velocità di modulazione}

\newpage
\section{Livello di Collegamento}

\subsection{Tipi di trasmissione}
\subsubsection{Sincrona}
\subsubsection{Asincrona}
\subsubsection{Orientata al carattere}
\subsubsection{Orientata al bit}

\subsection{Controlo degli errori}
\subsubsection{Ridondanza}

\subsection{Protocolli primario-secondario}
\subsubsection{RTS-CTS}
\subsubsection{XON-XOF}
\subsubsection{ARQ}


\newpage
\section{Livello di Rete}

\subsection{Terminologia}
\subsubsection{Rete}
\subsubsection{DTE}
\subsubsection{DCE}
\subsubsection{CPE}
\subsection{Tipologie di rete}
\subsection{Topologia delle reti}
\subsection{Qualità della rete}
\subsection{Routing}
\subsubsection{Tabella di routing}
\code{netstat -nr}
\subsection{Protocolli di Routing}
Claudio
\newpage
\section{Livello di Trasporto}
\newpage
\section{Livelli Applicativi}
\subsection{Servizi di Rete}
Tommaso
\subsubsection{Telnet}
\end{document}
