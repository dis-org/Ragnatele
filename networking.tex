\documentclass[a4paper,11pt]{paper}
\author{Quelli della B1}
\usepackage[utf8]{inputenc}
\usepackage[italian]{babel}
\usepackage{graphicx}
\usepackage[colorlinks=true,linkcolor=blue]{hyperref}
\graphicspath{{./images/}}

\def\code#1{\texttt{#1}}

\title{Networking}

\begin{document}

\maketitle
\newpage
\tableofcontents
\newpage

\section{Introduzione}
Gioara
\subsection{Terminologia}
\subsubsection{Tipi di flusso trasmissivo} 
\subsection{Il modello di riferimento ISO/OSI}
\subsection{Internet protocol suite (TCP/IP)}

\newpage

\section{Livello Fisico}
Nonostante l'amministratore di rete non abbia la possibilità di influirvi direttamente, è importante descrivere lo strato fisico poiché esso influenza significativamente le prestazioni della rete.

\subsection{Terminologia}
\subsubsection{Informazione} 
L'informazione è una grandezza misurabile in bit. In particolare, \[Q=log_{2}m\] dove $Q$ è il numero di bit necessari per rappresentare l'informazione relativa ad $m$ possibili stati. 

\subsubsection{Codice}
Al fine di rappresentare l'informazione in maniera tale da renderne più semplice la gestione, un codice associa sequenze di bit a caratteri. I codici che godono della più ampia diffusione sono:
\begin{itemize}
\item ASCII (American Standard Code for Information Interchange, 7 bit estesi a 1 byte)
\item BCD (Binary-Coded Decimal)
\item AIKEN 
\item Gray
\item EBCDIC (Extended Binary Coded Decimal Code, 8 bit), in uso presso le banche
\end{itemize}

\subsubsection{Segnale}
Si dice \textit{segnale} una grandezza fisica variabile nel tempo corrispondente un'informazione. Un segnale \textbf{analogico} varia in modo continuo nel tempo ed ha infiniti livelli di intensità; un segnale \textbf{digitale} varia invece in modo discreto e ha solo due livelli di intensità. Ogni tipo di dato può essere rappresentato in entrambe le maniere e può essere convertito da analogico a digitale e viceversa.
\\\\Fra i segnali analogici assumono particolare rilevanza i \textbf{segnali sinusoidali}, ossia segnali che variano nel tempo secondo una legge del tipo \[u=Usen(\omega t+\Phi )\] dove 
\begin{itemize}
\item $u$ è l'ampiezza istantanea
\item $U$ è l'ampiezza massima
\item $\omega $ è la velocità angolare %va spiegato meglio
\item $\Phi $ è lo sfasamento rispetto all'origine
\item l'intervallo di tempo impiegato dall'onda per tornare allo stesso livello d'intensità è detto \textit{periodo}.
\item $1/t=f$ è detta \textit{frequenza} (misurabile in Hz)\\\\
\end{itemize}
%la spiegazione trigonometrica va inclusa? 
\includegraphics[scale=0.5]{segnali_sin.png}

\subsubsection{Lunghezza d'onda}
In un segnale sinusoidale, la distanza tra due massimi relativi è detta \textit{lunghezza d'onda} $\lambda =c/f$ (dove $c$ è la velocità di propagazione del segnale).

\subsubsection{Spettro}
Lo spettro è l'insieme delle frequenze che compongono un segnale. Questa affermazione, non necessariamente di immediata comprensione, diventa subito chiara se si tiene presente il \textbf{teorema di Fourier}, il quale afferma che un segnale può essere rappresentato come somma di sinusoidi (potenzialmente infinite) con caratteristiche differenti.

\subsubsection{Ampiezza di banda}
L'ampiezza di banda è costituita dall'insieme di frequenze dello spettro \textit{effettivamente utilizzate} e corrisponde alla massima velocità teorica della rete. Si parla di \textit{banda larga} nel caso in cui l'ampiezza di banda sia sensibilmente superiore a quella utilizzata correntemente per le comunicazioni telefoniche.

\subsection{Qualità delle trasmissioni}
Come già accennato in precedenza, lo strato fisico è determinante per la qualità delle trasmissioni
\subsubsection{Ritardo}
\subsubsection{Tempo di risposta}
\subsubsection{Throughput}
\subsubsection{Latenza}
\subsubsection{Jitter}
\subsection{Filtri}
Un filtro è un sistema che tratta le varie componenti del segnale in modo diverso a seconda della loro frequenza.
\\E' opportuna innanzitutto una distinzione tra filtri \textit{passivi} ed \textit{attivi}: i primi sono costituiti solamente da resistenze e condensatori, mentre i secondi includono altre componenti, come i transistor e gli amplificatori. Inoltre, a seconda del comportamento, si distinguono quattro tipi di filtri:
\begin{itemize}
\item \textbf{filtro passa basso}: permette il passaggio delle frequenze al di sotto di una determinata \textit{frequenza di taglio}, definita come \[\frac{v_{out}}{v_{in}}=\frac{1}{(2)^{1/2}}\]
dove $v_{in}$ è il segnale in ingresso e $v_{out}$ il segnale in uscita.
\item \textbf{filtro passa alto}: complementare al filtro passa basso, permette il passaggio delle frequenze al di sopra della frequenza di taglio, definita come sopra
\item \textbf{filtro passa banda}: composizione di un filtro passa basso e un filtro passa alto
\item \textbf{filtro elimina banda}: complemento del filtro passa banda, blocca le frequenze comprese tra due frequenze di taglio.
\end{itemize}

\subsection{Modulazione}
\subsubsection{Ad onda continua}
\subsubsection{Impulsiva}
\subsubsection{Digitale}

\subsection{Alterazioni del segnale}
\subsubsection{Attenuazione}
\subsubsection{Distorsione}
\subsubsection{Rumore}
\subsubsection{Interferenza}

\subsection{Limiti alla velocità di trasferimento}
\subsubsection{Classificazione dei canali trasmissivi}
\subsubsection{Teorema di Nyquist}
\subsubsection{Teorema di Shannon}
\subsubsection{Velocità di modulazione}

\newpage
\section{Livello di Collegamento}

\subsection{Tipi di trasmissione}
\subsubsection{Sincrona}
\subsubsection{Asincrona}
\subsubsection{Orientata al carattere}
\subsubsection{Orientata al bit}

\subsection{Controllo degli errori}
\subsubsection{Ridondanza}

\subsection{Protocolli primario-secondario}
\subsubsection{RTS-CTS}
\subsubsection{XON-XOF}
\subsubsection{ARQ}


\newpage
\section{Livello di Rete}

\subsection{Terminologia}
\subsubsection{Rete}
\subsubsection{DTE}
\subsubsection{DCE}
\subsubsection{CPE}
\subsection{Tipologie di rete}
\subsection{Topologia delle reti}
\subsection{Qualità della rete}
\subsection{Routing}
\subsubsection{Tabella di routing}
\code{netstat -nr}
\subsection{Protocolli di Routing}
Claudio
\newpage
\section{Livello di Trasporto}
\newpage
\section{Livelli Applicativi}
\subsection{Servizi di Rete}
Tommaso
\subsubsection{Telnet}
\end{document}
