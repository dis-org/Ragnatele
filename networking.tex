\documentclass[a4paper,11pt]{paper}
\usepackage[utf8]{inputenc}
\usepackage[italian]{babel}
\usepackage{nameref}
\usepackage{graphicx}
\usepackage[colorlinks=true,linkcolor=blue]{hyperref}
\graphicspath{./images/}

\def\code#1{\texttt{#1}}
\def\sec#1{\section{#1}\label{#1}}
\def\sub#1{\subsection{#1}\label{#1}}
\def\subsub#1{\subsubsection{#1}\label{#1}}
\def\vedi#1{\nameref{#1}}

\title{Networking}

\begin{document}

\maketitle
\tableofcontents
\newpage

\sub{Introduzione}
Gioara
\subsub{I protocolli}
\subsub{Il modello ISO OSI}
\subsubsection{Internet protocol suit (TCP/IP)}

\newpage

\sec{Livello Fisico}
Arianna

\sub{Terminologia}
\subsub{Informazione}
\subsub{Codice}
\subsub{Segnale}
\subsub{Lunghezza d'onda}
\subsub{Spettro}
\subsub{Banda}
lalalalalalal \vedi{Codice} lalalall
\sub{Filtri}

\sub{Flusso di trasmissione}
\subsub{Simplex}
\subsub{Half-Duplex}
\subsub{Full-Duplex}

\sub{Modulazione}
\subsub{Ad onda continua}
\subsub{Impulsiva}
\subsub{Digitale}

\subsection{Qualità delle trasmissioni} non so se mi piace qui
\subsub{Ritardo}
\subsub{Tempo di risposta}
\subsub{Throughput}
\subsub{Latenza}
\subsub{Jitter}

\sub{Alterazioni del segnale}
\subsub{Attenuazione}
\subsub{Distorzione}
\subsub{Rumore}
\subsub{Interferenza}

\subsection{Limiti alla velocità di trasferimento}
\subsub{Classificazione dei canali trasmissivi}
\subsub{Teorema di Nyquist}
\subsub{Teorema di Shannon}
\subsubsection{Velocità di modulazione}

\newpage
\sec{Livello di Collegamento}

\sub{Tipi di trasmissione}
\subsub{Sincrona}
\subsub{Asincrona}
\subsub{Orientata al carattere} forse non è collegamento
\subsub{Orientata al bit}

\sub{Controlo degli errori}
\subsub{Ridondanza}

\subsection{Protocolli primario-secondario}
\subsub{RTS-CTS}
\subsub{XON-XOF}
\subsub{ARQ}

\sub{Protocolli internet}
\subsub{ARP}
\subsub{RARP}
\subsub{NDP}
\subsub{MAC}

\sub{Ethernet}

\newpage
\sec{Livello di Rete}

\sub{Terminologia}
\subsub{Rete}
\subsub{DTE}
\subsub{DCE}
\subsub{CPE}
\sub{Tipologie di rete}
\sub{Topologia delle reti}
\subsection{Qualità della rete}
\sub{Routing}
\subsub{Tabella di routing}
\code{netstat -nr}

\sub{Internet Protocol (IP)}


\sub{Protocolli di routing dinamico}
\subsub{ICMP}
\subsub{IGMP}
\subsub{RIP}
\subsub{OSPF}


\textbf{Claudio} Pannacci
\newpage
\sec{Livello di Trasporto}
\sub{Protocolli}
\subsub{TCP}
\subsub{UDP}
\newpage

\sec{Livello Applicazione}
\sub{Servizi di Rete}
Tommaso
\subsub{Telnet}
\subsub{FTP}
\subsub{SSH}
\subsub{BGP}
\subsub{DHCP}
\subsub{DNS}
\subsub{HTTP}

\end{document}
